% STAT451 HW6

\documentclass{article}
\usepackage{anysize}
\usepackage{amsmath}
\usepackage{graphicx}

\marginsize{2cm}{2cm}{2cm}{2cm}

\title{STAT451 HW6}
\author{Russell Miller}
\date{\today}

\begin{document}

\maketitle

% 4.39, 4.41, 4.42, 4.44, 4.45, 4.48, 4.51, 4.59, 4.60, 4.63, 4.67, 4.69, 4.73, |
% 4.77, 4.81, 4.91

\paragraph{4.39 The total number of hours, in units of 100 hours, that a family 
runs a vacuum cleaner over a period of one year is a random variable $X$ having 
the density function given in Exercise 4.15 on page 114. Find the variance of 
$X$.}
\begin{eqnarray*}
f(x) = \left\{\begin{array}{ll}
	x,& 0<x<1,\\
	2-x,& 1\leq x<2,\\
	0,& \mbox{elsewhere}
	\end{array}\right.
\end{eqnarray*}
Exercise 4.15 asked for the average number of hours per year that families run 
their vacuum cleaners. This is also known as $E[X]$. We found that
\begin{eqnarray*}
E[X] = 1
\end{eqnarray*}
The other piece we will need is $E[X^2]$. By the law of the unconscious 
statistician, we can let $g(X)=X^2$ and find $E[g(X)]$.
\begin{eqnarray*}
E[g(X)]=E[X^2] & = & \int_0^1 (x^2)xdx + \int_1^2 (x^2)(2-x)dx\\
	& = & \int_0^1 x^3dx + \int_1^2 (2x^2-x^3)dx\\
	& = & \left.\frac{x^4}{4}\right|_0^1 + \left.\left(\frac{2x^3}{3}-
				\frac{x^4}{4}\right)\right|_1^2\\
	& = & \frac{1}{4}+\left(\frac{16}{3}-4\right)-
				\left(\frac{2}{3}-\frac{1}{4}\right)\\
	& = & \frac{1}{4}+\frac{4}{3}-\frac{5}{12}\\
	& = & \frac{7}{6}
\end{eqnarray*}
And we have the variance
\begin{eqnarray*}
V[X] & = & E[X^2]-E[X]^2\\
	& = & \frac{7}{6}-1\\
	& = & \boxed{\frac{1}{6}}
\end{eqnarray*}

\pagebreak

\paragraph{4.41 Find the standard deviation of the random variable 
$g(X)=(2X+1)^2$ in Exercise 4.17 on page 114.}
\begin{center}\begin{tabular}{c | c c c}
$x$    & -3  & 6   & 9\\
\hline
$f(x)$ & 1/6 & 1/2 & 1/3
\end{tabular}\end{center}
Exercise 4.17 asked for $\mu_{g(x)}$. We found that it was
\begin{eqnarray*}
\mu_{g(x)} = E[g(X)] & = & \sum_{x=-3}^9 (2x+1)^2f(x)\\
	& = & (25)\left(\frac{1}{6}\right)+(169)\left(\frac{1}{2}\right)+
			(361)\left(\frac{1}{3}\right)\\
	& = & \frac{25+507+722}{6}\\
	& = & 209
\end{eqnarray*}
We're also going to need $E[(g(X))^2]$.
\begin{eqnarray*}
E[(g(X))^2] & = & (625)\left(\frac{1}{6}\right)+
			(28561)\left(\frac{1}{2}\right)+
			(130321)\left(\frac{1}{3}\right)\\
	& = & 57825
\end{eqnarray*}
Now we have the variance
\begin{eqnarray*}
\sigma^2=V[g(X)] & = & E[(g(X))^2]-E[g(X)]^2\\
	& = & 57825 - 209^2\\
	& = & 14144
\end{eqnarray*}
And the standard deviation is
\begin{eqnarray*}
\sigma & = & \sqrt{\sigma^2}\\
	& = & \boxed{118.9}
\end{eqnarray*}

\paragraph{4.42 Using the results of Exercise 4.21 on page 114, find the 
variance of $g(X)=X^2$, where $X$ is a random variable having the density 
function given in Exercise 4.12 on page 113.}
\begin{eqnarray*}
f(x)=\left\{\begin{array}{ll}
	2(1-x), & 0<x<1,\\
	0, & \mbox{elsewhere}
	\end{array}\right.
\end{eqnarray*}
The solution of 4.21 was the average, or $E[g(X)]$.
\begin{eqnarray*}
E[g(X)] = E(X^2) & = & \int_0^1 (x^2)2(1-x)dx\\
	& = & \int_0^1 2(x^2-x^3)dx\\
	& = & \frac{1}{6}
\end{eqnarray*}
We need to square that.
\begin{eqnarray*}
E[g(X)^2] = E(X^4) & = & \int_0^1 (x^4)2(1-x)dx\\
	& = & \int_0^1 2(x^4-x^5)dx\\
	& = & \frac{1}{15}
\end{eqnarray*}
Now we can calculate variance.
\begin{eqnarray*}
V[g(X)] & = & E[g(X)^2]-E[g(X)]^2\\
	& = & \frac{1}{15}-\left(\frac{1}{6}\right)^2\\
	& = & \boxed{\frac{7}{180} \approx .0389}
\end{eqnarray*}

%                                                                               |

\paragraph{4.44 Find the covariance of the random variables $X$ and $Y$ of 
Exercise 3.39 on page 101.\\}
Here's the table from the joint probability function $f(x,y)$, adding in the 
marginal values.
\begin{center}
\begin{tabular}{r|c c c c|c}
          & & x & & &\\
 $f(x,y)$ & 0 & 1 & 2 & 3 & $f_2(y)$\\
\hline
       0 & 0 & $\frac{3}{70}$ & $\frac{9}{70}$ & $\frac{3}{70}$ & 
       													    $\frac{15}{70}$\\
       &&&&\\
y$\;\;\;$1 & $\frac{2}{70}$ & $\frac{18}{70}$ & $\frac{18}{70}$ & 
											$\frac{2}{70}$ & $\frac{40}{70}$\\
       &&&&\\
       2 & $\frac{3}{70}$ & $\frac{9}{70}$ & $\frac{3}{70}$ & 0 & 
       													     $\frac{15}{70}$\\
	   &&&&\\
\hline
$f_1(x)$ & $\frac{5}{70}$ & $\frac{30}{70}$ & $\frac{30}{70}$ & $\frac{5}{70}$ &
																		1
\end{tabular}
\end{center}
Now the first thing we'll need is $\mu_X$
\begin{eqnarray*}
\mu_X = \sum_{x=0}^3 xf_1(x) = (0)\left(\frac{5}{70}\right)+
							(1)\left(\frac{30}{70}\right)+
							(2)\left(\frac{30}{70}\right)+
							(3)\left(\frac{5}{70}\right)
							= \frac{3}{2}
\end{eqnarray*}
And, of course, $\mu_Y$
\begin{eqnarray*}
\mu_Y = \sum_{y=0}^2 yf_2(y) = (0)\left(\frac{15}{70}\right)+
							(1)\left(\frac{40}{70}\right)+
							(2)\left(\frac{15}{70}\right)
							= 1
\end{eqnarray*}
We also need $E[XY]$
\begin{eqnarray*}
E[XY] = \sum_{x=0}^3\sum_{y=0}^2 xyf(x,y) & = & (0)(1)f(0,1)+
				(0)(2)f(0,2) + (1)(0)f(1,0) +\\
				& & (1)(1)f(1,1) + (1)(2)f(1,2) + (2)(0)f(2,0) +\\
				& & (2)(1)f(2,1) + (2)(2)f(2,2) + (3)(0)f(3,0) +\\
				& & (3)(1)f(3,1) = \frac{9}{7}
\end{eqnarray*}
And we put it all together to get the covariance.
\begin{eqnarray*}
\sigma_{XY} = E[XY] - \mu_X\mu_Y = \frac{9}{7}-\left(\frac{3}{2}\right)(1)
	= \boxed{-\frac{3}{14}}
\end{eqnarray*}

\paragraph{4.45 Find the covariance of the random variables $X$ and $Y$ of 
Exercise 3.49 on page 102.\\}
Alright, here goes $f(x,y)$, $f_1(x)$, $f_2(y)$, $\mu_X$, $\mu_Y$, $E[XY]$.
\begin{center}
\begin{tabular}{r|c c c|c}
           &      &  x   &      &\\
$f(x,y)$   &  1   &  2   &  3   & $f_2(y)$\\
\hline
         1 & 0.05 & 0.05 & 0.1  & 0.2\\
y$\;\;\;$2 & 0.05 & 0.1  & 0.35 & 0.5\\
         3 & 0    & 0.2  & 0.1  & 0.3\\
\hline
$f_1(x)$   & 0.1  & 0.35 & 0.55 & 1
\end{tabular}
\end{center}
\begin{eqnarray*}
\mu_X & = & \sum_{x=1}^3 xf_1(x) = (1)(.1) + (2)(.35) + (3)(.55) = 2.45\\
\mu_Y & = & \sum_{y=1}^3 yf_2(y) = (1)(.2) + (2)(.5) + (3)(.5) = 2.1\\
E[XY] & = & \sum_{x=1}^3\sum_{y=1}^3 xyf(x,y) = (1)(1)f(1,1) + (1)(2)f(1,2)+
				(1)(3)f(1,3) +\\
				& & (2)(1)f(2,1) + (2)(2)f(2,2) + (2)(3)f(2,3) +\\
				& & (3)(1)f(3,1) + (3)(2)f(3,2) + (3)(3)f(3,3)\\
				& = & 5.15\\
\sigma_{XY} & = & E[XY] - \mu_X\mu_Y = 5.15-(2.45)(2.1) = \boxed{0.005}
\end{eqnarray*}

\paragraph{4.48 Given a random variable $X$, with standard deviation $\sigma_X$ 
and a random variable $Y = a+bX$, show that if $b<0$, the correlation 
coefficient $\rho_{XY} = -1$, and if $b>0$, $\rho_{XY} = 1$.}
\begin{eqnarray*}
\rho_{XY} = \frac{\sigma_{XY}}{\sigma_X\sigma_Y} & = & 
	\frac{E[XY]-E[X]E[Y]}{\sqrt{E[X^2]-E[X]}\sqrt{E[Y^2]-E[Y]}}\\
	& = & \frac{E[X(a+bX)]-E[X]E[a+bX]}{\sqrt{E[X^2]-E[X]}
					\sqrt{E[(a+bX)^2]-E[a+bX]}}\\
	& = & \frac{E[Xa+bX^2]-bE[X]^2}{V[X]^2\sqrt{E[a^2+2abX+b^2X^2]-bE[X]}}\\
	& = & \frac{aE[X]+bE[X^2]-bE[X]^2}{V[X]^2\sqrt{2abE[X]+b^2E[X^2]-bE[X]}}\\
	& = & \frac{aE[X]+bV[X]}{V[X]^2\sqrt{b(2aE[X]+bE[X^2]-E[X])}}
\end{eqnarray*}
I'm dizzy, I give up.

%                                                                               |

\paragraph{4.51 Referring to Exercise 4.35 on page 122, find the mean and 
variance of the discrete random variable $Z=3X-2$, when $X$ represents the 
number of errors per 100 lines of code.}
\begin{center}\begin{tabular}{c|c c c c c}
$x$    & 2    & 3    & 4   & 5   & 6\\
\hline
$f(x)$ & 0.01 & 0.25 & 0.4 & 0.3 & 0.04
\end{tabular}\end{center}
\begin{eqnarray*}
\mu_Z = E[3X-2] & = & \sum_{x=2}^6 (3x-2)f(x)\\
	& = & (4\times0.01) + (7\times0.25) + (10\times0.4) + (13\times0.3) + 
		(16\times0.04)\\
	& = & \boxed{10.33} \leftarrow\mbox{ mean}\\
E[(3X-2)^2] & = & \sum_{x=2}^6 (3x-2)^2f(x)\\
	& = & (16\times0.01) + (49\times0.25) + (100\times0.4) + (169\times0.3) + 
		(256\times0.04)\\
	& = & 113.35\\
\sigma_Z^2 = E[Z^2]-E[Z]^2 & = & 113.35-10.33^2\\
	& = & \boxed{6.6411} \leftarrow\mbox{ variance}
\end{eqnarray*}

\pagebreak

\paragraph{4.59 Use Theorem 4.7 to evaluate $E[2XY^2-X^2Y]$ for the joint 
probability distribution shown in Table 3.1 on page 92.}
\begin{center}
Table 3.1 Joint Probability Distribution\\
\begin{tabular}{r|c c c|c}
& & $x$ &\\
$f(x,y)$ & 0 & 1 & 2 & $f_2(y)$\\
\hline
0 & $\frac{3}{28}$ & $\frac{9}{28}$ & $\frac{3}{28}$ & $\frac{15}{28}$\\
 & & & &\\
$y\;\;\;\;1$ & $\frac{3}{14}$ & $\frac{3}{14}$ & 0 & $\frac{3}{7}$\\
 & & & &\\
2 & $\frac{1}{28}$ & 0 & 0 & $\frac{1}{28}$\\
 & & & &\\
\hline
$f_1(x)$ & $\frac{5}{14}$ & $\frac{15}{28}$ & $\frac{3}{28}$ & 1
\end{tabular}\end{center}
By Theorem 4.7 we see that we can rewrite this as
\begin{eqnarray*}
E[2XY^2-X^2Y] = E[2XY^2]-E[X^2Y]
\end{eqnarray*}
First we'll find $E[2XY^2]$
\begin{eqnarray*}
E[2XY^2] = \sum_{x=0}^2\sum_{y=0}^2 2xy^2f(x,y) = 
	2(0)(0^2)\left(\frac{3}{28}\right) + 2(0)(1^2)\left(\frac{3}{14}\right) +\\ 
	2(0)(2^2)\left(\frac{1}{28}\right) + 2(1)(0^2)\left(\frac{9}{28}\right) +
	2(1)(1^2)\left(\frac{3}{14}\right) + 2(2)(0^2)\left(\frac{3}{28}\right)
	= \frac{3}{7}
\end{eqnarray*}
Next $E[X^2Y]$
\begin{eqnarray*}
E[X^2Y] = \sum_{x=0}^2\sum_{y=0}^2 x^2yf(x,y) = 
	(0^2)(0)\left(\frac{3}{28}\right) + (0^2)(1)\left(\frac{3}{14}\right) +\\
	(0^2)(2)\left(\frac{1}{28}\right) + (1^2)(0)\left(\frac{9}{28}\right) +
	(1^2)(1)\left(\frac{3}{14}\right) + (2^2)(0)\left(\frac{3}{28}\right)
	= \frac{3}{14}
\end{eqnarray*}
And the grande finale
\begin{eqnarray*}
E[2XY^2]-E[X^2Y] = \frac{3}{7}-\frac{3}{14}=\boxed{\frac{3}{14}}
\end{eqnarray*}

\paragraph{4.60 Seventy new jobs are opening up at an automobile manufacturing 
plant, but 1000 applicants show up for the 70 positions. To select the best 70 
from among the applicants, the company gives a test that covers mechanical 
skill, manual dexterity, and mathematical ability. The mean grade on this test 
turns out to be 60, and the scores have a standard deviation 6. Can a person 
who has an 84 score count on getting one of the jobs? [$Hint$: Use Chebyshev's 
theorem.] Assume that the distribution is symmetric about the mean.\\}
We can see that someone who scores an 84 is 4 standard deviations above the 
mean grade. By Chebyshev's Theorem we know that
\begin{eqnarray*}
P(\mu-k\sigma<X<\mu+k\sigma) & \geq & 1-\frac{1}{k^2}\\
P(60-4\times6<X<60+4\times6) & \geq & 1-\frac{1}{4^2}
\end{eqnarray*}
But the part we care about is
\begin{eqnarray*}
P(X<84) \geq 0.9375
\end{eqnarray*}
So scoring an 84 means only 0.0625 of the scores are as good. So because the top
0.07 of the applicants are going to get a job, you could certainly count on 
getting one.

%                                                                               |

\paragraph{4.63 Suppose that you roll a fair 10-sided die (0,1,2,...,9) 500 
times. Using Chebyshev's theorem, compute the probability that the sample mean, 
$X$ is between 4 and 5.\\}
I've never seen a die with a zero, but any way...
\begin{eqnarray*}
\mu & = & \sum_{x=0}^9 xf(x) = (45)\left(\frac{1}{10}\right) = 4.5\\
E[X^2] & = & \sum_{x=0}^9 x^2f(x) = (285)\left(\frac{1}{10}\right) = 28.5\\
\sigma^2 & = & 28.5-(4.5)^2 = 8.25\\
\sigma & = & \sqrt{\sigma} \approx 2.87
\end{eqnarray*}
Now we're ready to use Chebyshev's theorem.
\begin{eqnarray*}
P(\mu-k\sigma<X<\mu+k\sigma) & \geq & 1-\frac{1}{k^2}\\
P(4.5-k2.87<X<4.5+k2.87) & \geq & 1-\frac{1}{k^2}
\end{eqnarray*}
In order to set up the inequality, we let $k=0.1742$.
\begin{eqnarray*}
P(4<X<5) & \geq & 1-\frac{1}{0.0303}\\
	& \geq & -32.003
\end{eqnarray*}
Horrible bound. I did something wrong.

\paragraph{4.67 A random variable $X$ has a mean $\mu=10$ and a variance $
\sigma^2=4$. Using Chebyshev's theorem, find}
\begin{enumerate}
\item[a.] $P(|X-10|\geq 3)$
\begin{eqnarray*}
P(|X-10|\geq 3) = 1-P(|X-10|<3)=1-P(-3<X-10<3)=1-P(10-3<X<10+3)
\end{eqnarray*}
$k\sigma=3$, so $k=3/2$
\begin{eqnarray*}
1-P(10-(3/2)(2)<X<10+(3/2)(2))\boxed{\leq 4/9}
\end{eqnarray*}

\item[b.] $P(|X-10|<3)$
\begin{eqnarray*}
P(10-(3/2)(2)<X<10+(3/2)(2)) & \geq & 1-\frac{4}{9}\\
				\boxed{\geq \frac{5}{9}}
\end{eqnarray*}

\item[c.] $P(5<X<15)$\\
Given $\mu=10$ and $\sigma=2$, $10-2k=5$ and $10+2k=15$. Thus $k=5/2$.
\begin{eqnarray*}
P(10-(5/2)(2)<X<10+(5/2)(2)) & \geq & 1-(4/25)\\
				\boxed{\geq \frac{21}{25}}
\end{eqnarray*}

\item[d.] the value of the constant $c$ such that $P(|X-10|\geq c)\leq 0.04$
\begin{eqnarray*}
P(|X-10|\geq c) & = & 1-P(|X-10|<c)=1-P(-c<X-10<c)=1-P(10-c<X<10+c)\\
0.04 & = & \frac{1}{k^2}\\
	k & = & 5\\
	c & = & 2k\\
	\boxed{c=10}
\end{eqnarray*}
\end{enumerate}

\paragraph{4.69 Let $X$ represent the number that occurs when a red die is 
tossed and $Y$ the number that occurs when a green die is tossed. Find}
\begin{enumerate}
\item[a.] $E[X+Y]$
\begin{eqnarray*}
E[X+Y] = E[X]+E[Y] & = & \sum_{x=1}^6 xf(x)+\sum_{y=1}^6 yf(y)\\
	& = & (21)\left(\frac{1}{6}\right)+(21)\left(\frac{1}{6}\right)\\
	& = & \boxed{7}
\end{eqnarray*}

\item[b.] $E[X-Y]$
\begin{eqnarray*}
E[X-Y] = E[X]-E[Y] & = & \sum_{x=1}^6 xf(x)-\sum_{y=1}^6 yf(y)\\
	& = & (21)\left(\frac{1}{6}\right)-(21)\left(\frac{1}{6}\right)\\
	& = & \boxed{0}
\end{eqnarray*}
	
\item[c.] $E[XY]$
\begin{eqnarray*}
E[XY] & = & \sum_{x=1}^6\sum_{y=1}^6 xyf(x,y)\\
	& = & (21)(21)\left(\frac{1}{36}\right)\\
	& = & \boxed{12.25}
\end{eqnarray*}

\end{enumerate}

%                                                                               |

\paragraph{4.73 Consider a random variable $X$ with density function}
\begin{eqnarray*}
f(x)=\left\{\begin{array}{ll}
	\frac{1}{5},& 0\leq x\leq 5,\\
	0, & \mbox{elsewhere.}
	\end{array}\right.
\end{eqnarray*}
\begin{enumerate}
\item[a.] Find $\mu=E[X]$ and $\sigma^2 = E[(X-\mu)^2]$.
\begin{eqnarray*}
E[X] & = & \int_0^5 xf(x)dx = \int_0^5 \frac{x}{5}dx = \boxed{\frac{5}{2}}\\
E[(X-\mu)^2] & = & \int_0^5 (x-\mu)^2f(x)dx = 
	\int_0^5 \left(x-\frac{5}{2}\right)^2\left(\frac{1}{5}\right)dx = 
		\boxed{\frac{25}{12}}
\end{eqnarray*}

\item[b.] Demonstrate that Chebyshev's theorem holds for $k=2$ and $k=3$.\\
When $k=2$
\begin{eqnarray*}
P\left(\frac{5}{2}-2(2.89)<X<\frac{5}{2}+2(2.89)\right) \geq \frac{3}{4}\\
P(-.387<X<5.387) \geq \frac{3}{4}
\end{eqnarray*}
This holds because $f(x)$ is nonzero when $0<X<5$, which lies within this range.\\
When $k=3$
\begin{eqnarray*}
P\left(\frac{5}{2}-3(2.89)<X<\frac{5}{2}+3(2.89)\right) \geq \frac{8}{9}\\
P(-1.83<X<6.83) \geq \frac{8}{9}
\end{eqnarray*}
This holds because $f(x)$ is nonzero when $0<X<5$, which lies within this range.

\end{enumerate}

\paragraph{4.77 The length of time $Y$ in minutes required to generate a human 
reflex to tear gas has density function}
\begin{eqnarray*}
f(y)=\left\{\begin{array}{ll}
	\frac{1}{4}e^{-y/4}, & 0\leq y< \infty\\
	0, & \mbox{elsewhere.}
	\end{array}\right.
\end{eqnarray*}
\begin{enumerate}
\item[a.] What is the mean time to reflex?
\begin{eqnarray*}
E[Y] = \int_0^\infty y\frac{1}{4}e^{-y/4}dy = \boxed{4}
\end{eqnarray*}

\item[b.] Find $E[Y^2]$ and $V[Y]$.
\begin{eqnarray*}
E[Y^2] = \int_0^\infty y^2\frac{1}{4}e^{-y/4}dy = \boxed{32}\\
V[Y] = E[Y^2]-E[Y]^2 = 32-16=\boxed{16}
\end{eqnarray*}

\end{enumerate}

\paragraph{4.81 Prove Chebyshev's theorem when $X$ is a discrete random variable.\\}
We can write variance as
\begin{eqnarray}
\sigma^2=E[(X-\mu)^2]=\sum_x(x-\mu)^2f(x)\\
	= \sum_{x<\mu-k\sigma} (x-\mu)^2f(x) + \sum_{x>=\mu+k\sigma} (x-\mu)^2f(x)\\
	\geq \sum_{x<\mu-k\sigma} k^2\sigma^2f(x) + \sum_{x>=\mu+k\sigma} k^2\sigma^2f(x)\\
	\sum_{x<\mu-k\sigma} f(x) + \sum_{x>=\mu+k\sigma} f(x) \leq \frac{1}{k^2}\\
	P(\mu-k\sigma<X<\mu+k\sigma)=\sum_{x=\mu-k\sigma}^{\mu+k\sigma} f(x) \geq 1-\frac{1}{k^2}
\end{eqnarray}

\paragraph{4.91 Consider the joint density function}
\begin{eqnarray*}
f(x,y)=\left\{\begin{array}{ll}
	\frac{16y}{x^3},& x>2,0<y<1,\\
	0,& \mbox{elsewhere.}
	\end{array}\right.
\end{eqnarray*}
\textbf{Compute the correlation coefficient $\rho_{XY}$.}\\

First we need the marginal density functions.
\begin{eqnarray*}
f_1(x) & = & \int_{-\infty}^{\infty} f(x,y)dy = \int_0^1 \frac{16y}{x^3}dy
	= \left.\frac{8y^2}{x^3}\right|_0^1\\
	& = & \frac{8}{x^3}, x>2\\
f_2(y) & = & \int_{-\infty}^{\infty} f(x,y)dx = \int_2^\infty \frac{16y}{x^3}dx
	= \left.\frac{-8y}{x^2}\right|_2^\infty\\
	& = & 2y, 0<y<1
\end{eqnarray*}

Next we need $\mu_X$ and $\mu_Y$.
\begin{eqnarray*}
\mu_X & = & \int_2^\infty \frac{8}{x^2}dx = 4\\
\mu_Y & = & \int_0^1 2y^2dy = \frac{2}{3}
\end{eqnarray*}

Now we need $E[XY]$.
\begin{eqnarray*}
E[XY] = \int_2^\infty \int_0^1 \frac{16y^2}{x^2}dxdy = \frac{8}{3}
\end{eqnarray*}

And finally we can compute $\sigma_{XY}$.
\begin{eqnarray*}
\sigma_{XY} & = & E[XY]-\mu_X\mu_Y = \frac{8}{3}-(4)\left(\frac{2}{3}\right)\\
	 & = & \boxed{0}
\end{eqnarray*}

\end{document}
