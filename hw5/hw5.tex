% STAT451 HW5

\documentclass{article}
\usepackage{anysize}
\usepackage{amsmath}
\usepackage{amssymb}
\usepackage{graphicx}

\marginsize{2cm}{2cm}{2cm}{2cm}

\title{STAT451 HW5}
\author{Russell Miller}
\date{\today}

\begin{document}

\maketitle

% 3.27, 3.29, 3.30, 3.36, 3.37, 3.39, 3.41, 3.51, 3.53, 3.55, and 3.56.         |

\paragraph{3.27 The time to failure in hours of an important piece of electronic 
equipment used in a manufactured DVD player has the density function}
\begin{eqnarray*}
f(x) = \left\{ \begin{array}{ll}
	\left(\frac{1}{2000}\right)e^{\frac{-x}{2000}}, & x\geq 0,\\
	0, & x<0.
	\end{array} \right.
\end{eqnarray*}
\begin{enumerate}
\item[\textbf{a.}] \textbf{Find $F(x)$.}
\begin{eqnarray*}
F(x) = \int f(x)dx = \left\{ \begin{array}{ll}
	0, & x<0\\
	-e^{\frac{-x}{2000}}, & x \geq 0
	\end{array} \right.
\end{eqnarray*}

\item[\textbf{b.}] \textbf{Determine the probability that the component (and 
thus the DVD player) lasts more than 1000 hours before the component needs to 
be replaced.}
\begin{eqnarray*}
P(X>1000) = \int_{1000}^\infty \frac{e^{\frac{-x}{2000}}}{2000}dx = .6065
\end{eqnarray*}

\item[\textbf{c.}] \textbf{Determine the probability that the component fails
before 2000 hours.}
\begin{eqnarray*}
P(X<2000) = \int_0^{2000} \frac{e^{\frac{-x}{2000}}}{2000}dx = .6321
\end{eqnarray*}
\end{enumerate}

\paragraph{3.29 An important factor in solid missile fuel is the particle size 
distribution. Significant problems occur if the particle sizes are too large. 
From production data in the past, it has been determined that the particle size 
(in micrometers) distribution is characterized by}
\begin{eqnarray*}
f(x) = \left\{ \begin{array}{ll}
	3x^{-4}, & x>1,\\
	0, & \mbox{elsewhere}
	\end{array} \right.
\end{eqnarray*}
\begin{enumerate}
\item[\textbf{a.}] \textbf{Verify that this is a valid density function.\\}
From my notes, a Probability Density Function is any function $f(x)$ that
satisfies:
\begin{itemize}
\item $f(x) \geq 0$ for all real x,
\item $\int_{-\infty}^\infty f(x)dx = 1$,
\item For event A\\
	$P(x \in A) = \int_{x \in A} f(x)dx$
\end{itemize}
The first bullet is satisfied, there are no negative values of $f(x)$.
The second bullet can be verified by
\begin{eqnarray*}
\int_1^\infty f(x)dx = \frac{-1}{\infty} - \frac{-1}{1^4} = 1
\end{eqnarray*}
Since we're not given any events or probabilities, this is sufficient.

\pagebreak
\item[\textbf{b.}] \textbf{Evaluate $F(x)$.}
\begin{eqnarray*}
F(x) = \left\{ \begin{array}{ll}
	0, & x < 1\\
	-x^{-3}, & x \geq 1
	\end{array} \right.
\end{eqnarray*}

\item[\textbf{c.}] \textbf{What is the probability that a random particle 
from the manufactured fuel exceeds 4 micrometers?}
\begin{eqnarray*}
P(X>4) = \int_{4}^\infty 3x^{-4}dx = \frac{1}{64}
\end{eqnarray*}
\end{enumerate}

\paragraph{3.30 Measurements of scientific systems are always subject to
variation, some more than others. There are many structures for measurement 
error and statisticians spend a great deal of time modeling these errors. 
Suppose the measurement error $X$ of a certain physical quantity is decided 
by the density function}
\begin{eqnarray*}
f(x) = \left\{ \begin{array}{ll}
	k(3-x^2), & -1 \leq x \leq 1,\\
	0, & \mbox{elsewhere.}
	\end{array} \right.
\end{eqnarray*}
\begin{enumerate}
\item[\textbf{a.}] \textbf{Determine $k$ that renders $f(x)$ a valid density
function.\\}
In order for $f(x)$ to be a valid density function, the following must hold.
\begin{eqnarray*}
\int_{-1}^1 k(3-x^2)dx & = & 1\\
\int_{-1}^1 (3k-kx^2)dx & = & 1\\
\left. 3kx - \frac{x^3}{3k} \right|_{-1}^{1} & = & 1\\
\left. k(3x - \frac{x^3}{3}) \right|_{-1}^{1} & = & 1\\
\frac{8k}{3} - \frac{-8k}{3} & = & 1\\
\frac{16k}{3} & = & 1\\
k & = & \frac{3}{16}
\end{eqnarray*}

And we'll double check by plugging it in
\begin{eqnarray*}
\int_{-1}^1 \left(\frac{3}{16}\right)(3-x^2)dx & = & 1\\
& &  \blacksquare
\end{eqnarray*}
\end{enumerate}

\pagebreak

\paragraph{3.36 On a labratory assignment, if the equipment is working, the
density function of the observed outcome, $X$, is}
\begin{eqnarray*}
f(x)= \left\{ \begin{array}{ll}
	2(1-x), & 0<x<1,\\
	0, & \mbox{otherwise}.
	\end{array} \right.
\end{eqnarray*}
\begin{enumerate}
\item[\textbf{a.}] \textbf{Calculate $P(X \leq \frac{1}{3})$.}
\begin{eqnarray*}
P(X \leq \frac{1}{3}) & = & F(\frac{1}{3})\\
 & = & \int_{0}^{\frac{1}{3}} 2(1-x)dx\\
 & = & \left. 2x-x^2 \right|_{0}^{\frac{1}{3}}\\
 & = & \left(\frac{2}{3}-\frac{1}{9}\right) - 0\\
 & = & \frac{5}{9}
\end{eqnarray*}

\item[\textbf{b.}] \textbf{What is the probability that $X$ will exceed 0.5?}
\begin{eqnarray*}
P(X>0.5) & = & F(1)-F(0.5)\\
	& = & \int_{0.5}^{1} 2(1-x)dx\\
	& = & \left. 2x-x^2 \right|_{0.5}^{1}\\
	& = & 1 - \frac{3}{4}\\
	& = & \frac{1}{4}
\end{eqnarray*}

\item[\textbf{c.}] \textbf{Given that $X \geq 0.5$, what is the probability 
that $X$ will be less than 0.75?}
\begin{eqnarray*}
P(X<0.75 \mid X \geq 0.5) & = & \frac{P(0.5 \leq X < 0.75)}{P(X \geq 0.5)}
\end{eqnarray*}

Recall that part \textbf{b} asked for $P(X>0.5)$ which is the same as
$P(X \geq 0.5)$.
\begin{eqnarray*}
	& = & \frac{\int_{0.5}^{0.75} 2(1-x)dx}{\frac{1}{4}}\\
	& = & 4(\left. 2x-x^2 \right|_{0.5}^{0.75})\\
	& = & 4\left(\left(\frac{6}{4}-\frac{9}{16}\right)-
		\left(2-\frac{1}{4}\right)\right)\\
	& = & 4\left(\frac{15}{16}-\frac{3}{4}\right)\\
	& = & 4\left(\frac{3}{16}\right)\\
	& = & \frac{3}{4}
\end{eqnarray*}
\end{enumerate}

\pagebreak

\paragraph{3.37 Determine the values of $c$ so that the following functions 
represent joint probability distributions of the random variables $X$ and 
$Y$.}
\begin{enumerate}
\item[\textbf{a.}] $f(x,y)=cxy$, for $x=1,2,3$; $y=1,2,3$\\
Need the following to be true.
\begin{eqnarray*}
\sum\limits_x \sum\limits_y f(x,y) & = & 1
\end{eqnarray*}

Plugging in $f(x,y)$.
\begin{eqnarray*}
\sum\limits_{x=1}^3 \sum\limits_{y=1}^3 cxy & = & 1\\
c \sum\limits_{x=1}^3 \sum\limits_{y=1}^3 xy & = & 1\\
c((1)(1)+(1)(2)+(1)(3)+(2)(1)+(2)(2)+(2)(3)+(3)(1)+(3)(2)+(3)(3)) & = & 1\\
c & = & \frac{1}{36}
\end{eqnarray*}

\item[\textbf{b.}] $f(x,y)=c\mid x-y\mid$, for $x=-2,0,2$; $y=-2,3$\\
\begin{eqnarray*}
c(\mid-2-(-2)\mid+\mid-2-3\mid+\mid0-(-2)\mid+\mid0-3\mid+\mid2-(-2)\mid+
	\mid2-3\mid) & = & 1\\
c(0+5+2+3+4+1) & = & 1\\
c & = & \frac{1}{15}
\end{eqnarray*}
\end{enumerate}

\paragraph{3.39 From a sack of fruit containing 3 oranges, 2 apples, and 
3 bananas, a random sample of 4 pieces of fruit is selected. If $X$ is the 
number of oranges and $Y$ is the number of apples in the sample, find}
\begin{enumerate}
\item[\textbf{a.}] \textbf{the joint probability distribution of $X$ and 
$Y$}\\
There are ${8 \choose 4} = 70$ ways to take a random sample of 4 pieces of 
fruit from the sack containing a total of 8. This will be the denominator
for all probabilities in the table. Now we use the product rule to choose
the given number of each fruit.\\
For example, the ways to get 2 oranges and 2 apples is 
${3 \choose 2}{2 \choose 2} = 3$ and the probability of this happening is
therefore $\frac{3}{70}$.\\
For the case when $X=0$ and $Y=0$, this assumes the only remaining fruit
are bananas, but it's impossible to select 4 fruit because there are only
3 bananas. So the probability of this happening is 0.\\
Another type of case is where $0<X+Y<4$. We will need to borrow from the
bananas to complete these, by multiplying ${3 \choose x}{2 \choose y}
{3 \choose 4-(x+y)}$ and dividing that by 70.
\begin{center}
\begin{tabular}{|c|c|c c c c|}
\hline
   & & & x & &\\
\hline
   &   & 0 & 1 & 2 & 3\\
\hline
   & 0 & 0 & $\frac{3}{70}$ & $\frac{9}{70}$ & $\frac{3}{70}$\\
   &&&&&\\
 y & 1 & $\frac{2}{70}$ & $\frac{18}{70}$ & $\frac{18}{70}$ & $\frac{2}{70}$\\
   &&&&&\\
   & 2 & $\frac{3}{70}$ & $\frac{9}{70}$ & $\frac{3}{70}$ & 0\\
\hline
\end{tabular}
\end{center}

\pagebreak
\item[\textbf{b.}] \textbf{$P[(X,Y) \in A]$, where $A$ is the region that
is given by $\{(x,y) \mid x+y \leq 2\}$}\\
The values $(x,y)$ that satisfy this equation are 
(0,0),(0,1),(0,2),(1,0),(1,1),(2,0). 
Their values from the table are 
$0,\frac{3}{70},\frac{9}{70},\frac{2}{70},\frac{18}{70},\frac{3}{70}$.
The total probability for $A$ is the sum of these.
\begin{center}
$P[(X,Y) \in A] = \frac{35}{70} = \frac{1}{2}$
\end{center}
\end{enumerate}

\paragraph{3.41 A candy company distributes boxes of chocolates with a 
mixture of creams, toffees, and cordials. Suppose that the weight of 
each box is 1 kilogram, but the individual weights of the creams, 
toffees, and cordials vary from box to box. For a randomly selected box, 
let $X$ and $Y$ represent the weights of the creams and the toffees, 
respectively, and suppose that the joint density function of these 
variables is}
\begin{eqnarray*}
f(x,y)= \left\{ \begin{array}{lll}
	24xy, & 0 \leq x \leq 1, & 0 \leq y \leq 1,\\
	& x + y \leq 1,\\
	0, & \mbox{elsewhere.}
	\end{array} \right.
\end{eqnarray*}
\begin{enumerate}
\item[\textbf{a.}] \textbf{Find the probability that in a given box 
the cordials account for more than 1/2 of the weight.}
\begin{eqnarray*}
P(X+Y<\frac{1}{2}) & = & \int_0^{\frac{1}{2}} \int_0^{\frac{1}{2}-y} 24xydxdy\\
	& = & \int_0^{\frac{1}{2}} 24y \left( \int_0^{\frac{1}{2}-y} xdx \right)dy\\
	& = & \int_0^{\frac{1}{2}} 24y\frac{(2y-1)^2}{8}dy\\
	& = & 3 \int_0^{\frac{1}{2}} y(2y-1)^2dy\\
	& = & \frac{1}{16}
\end{eqnarray*}

\item[\textbf{b.}] \textbf{Find the marginal density for the weight
of the creams.}\\
Let $f_1(x)$ represent the marginal density for $X$.
\begin{eqnarray*}
f_1(x) & = & \int_{-\infty}^\infty f(x,y)dy\\
	& = & \int_0^{1-x} 24xydy\\
	& = & 24x \int_0^{1-x} ydy\\
	& = & 24x \frac{(x-1)^2}{2}\\
	& = & 12x(x-1)^2\\
f_1(x) & = & \left\{ \begin{array}{ll}
	12x(x-1)^2, & 0 \leq x \leq 1,\\
	0, & \mbox{elsewhere}
	\end{array} \right.
\end{eqnarray*}

\pagebreak

\item[\textbf{c.}] \textbf{Find the probability that the weight of 
the toffees in a box is less than 1/8 of a kilogram if it 
is known that creams constitute 3/4 of the weight.}\\
First we need to find $f(y\mid x)$.
\begin{eqnarray*}
f(y\mid x) & = & \frac{f(x,y)}{f_1(x)}\\
	& = & \frac{24xy}{12x(x-1)^2}\\
	& = & \frac{12y}{(x-1)^2}\\
P(Y<\frac{1}{8}\mid X=\frac{3}{4}) & = & \int_0^{\frac{1}{8}} 
		\frac{12y}{((\frac{3}{4})-1)^2}dy\\
	& = & \int_0^\frac{1}{8} 192ydy\\
	& = & \frac{3}{2}
\end{eqnarray*}
\end{enumerate}

\paragraph{3.51 Consider an experiment that consists of 2 rolls of a 
balanced die. If $X$ is the number of 4s and $Y$ is the number of 5s 
obtained in the 2 rolls of the die, find}
\begin{enumerate}
\item[\textbf{a.}] \textbf{the joint probability distribution of $X$ 
and $Y$;}\\
The probability of rolling 2 dice and not getting a 4 or a 5 on 
either roll is $\left(\frac{4}{6}\right)^2$ because of the other 4
possible rolls each time. To get a 4 but not a 5, or a 5 but not a 4
there are two ways: getting the 4 first (or 5), or second. The
probability of rolling a 4 is just $\frac{1}{6}$, then you cannot roll
a 5 or another 4, so multiply by $\frac{4}{6}$. Considering the reverse
order, you can just multiply this by 2. Getting the same roll twice
is just $\left(\frac{1}{6}\right)^2$. Getting one of each is also
$\left(\frac{1}{6}\right)^2$, but you multiply by 2 for the reverse
order.
\begin{center}
\begin{tabular}{|c|c|c c c|}
\hline
   & & & x &\\
\hline
   &   & 0 & 1 & 2\\
\hline
   & 0 & $\frac{16}{36}$ & $\frac{8}{36}$ & $\frac{1}{36}$\\
   &&&&\\
 y & 1 & $\frac{8}{36}$ & $\frac{2}{36}$ & 0\\
   &&&&\\
   & 2 & $\frac{1}{36}$ & 0 & 0\\
\hline
\end{tabular}
\end{center}

\item[\textbf{b.}] \textbf{$P[(X,Y)\in A]$, where $A$ is the region
$\{(x,y)\mid 2x+y<3\}$.}\\
The values $(x,y)$ that satisfy this equation are 
(0,0),(0,1),(0,2),(1,0). Their values from the table are
$\frac{16}{36},\frac{8}{36},\frac{1}{36},\frac{8}{36}$. Just need to
sum them.
\begin{center}
$P[(X,Y) \in A] = \frac{33}{36} = \frac{11}{12}$
\end{center}
\end{enumerate}

\pagebreak

\paragraph{3.53 Three cards are drawn without replacement from the 12 
face cards (jacks, queens, and kings) of an ordinary deck of 52 
playing cards. Let $X$ be the number of kings selected and $Y$ the 
number of jacks. Find}
\begin{enumerate}
\item[\textbf{a.}] \textbf{the joint probability distribution of $X$ 
and $Y$;}\\
The probability of drawing a queen on the first card is 
$\frac{4}{12}$. The probability of drawing a second queen from the
remaining cards is $\frac{3}{11}$, and a third $\frac{2}{10}$. This
product is the probability of getting no kings or jacks. There are 3
ways to get one jack and no kings, because you can get it on the 
first, second, or third draw. The probability of drawing it first is
$\frac{4}{12}$, and the probability of not getting a king or jack 
next, since there are still 4 kings and 3 jacks, is $\frac{4}{11}$.
This means it had to be a queen, of which there were 4. Now there are
3 for the final draw, resulting in $\frac{3}{10}$. This sort of
calculation will continue in the following table.
\begin{center}
\begin{tabular}{c|c|c}
Composition & Resulting Hand & Probability\\
\hline
0J/0K & QQQ & $\frac{4}{12}\times\frac{3}{11}\times\frac{2}{10}=\frac{1}{55}$\\
\hline
1J/0K & JQQ & $\frac{4}{12}\times\frac{4}{11}\times\frac{3}{10}=\frac{2}{55}$\\
	& QJQ & $\frac{4}{12}\times\frac{4}{11}\times\frac{3}{10}=\frac{2}{55}$\\
	& QQJ & $\frac{4}{12}\times\frac{3}{11}\times\frac{4}{10}=\frac{2}{55}$\\
\hline
2J/0K & JJQ & $\frac{4}{12}\times\frac{3}{11}\times\frac{4}{10}=\frac{2}{55}$\\
	& JQJ & $\frac{4}{12}\times\frac{4}{11}\times\frac{3}{10}=\frac{2}{55}$\\
	& QJJ & $\frac{4}{12}\times\frac{4}{11}\times\frac{3}{10}=\frac{2}{55}$\\
\hline
3J/0K & JJJ & $\frac{4}{12}\times\frac{3}{11}\times\frac{2}{10}=\frac{1}{55}$\\
\hline
0J/1K & KQQ & $\frac{4}{12}\times\frac{4}{11}\times\frac{3}{10}=\frac{2}{55}$\\
	& QKQ & $\frac{4}{12}\times\frac{4}{11}\times\frac{3}{10}=\frac{2}{55}$\\
	& QQK & $\frac{4}{12}\times\frac{3}{11}\times\frac{4}{10}=\frac{2}{55}$\\
\hline
1J/1K & JKQ & $\frac{4}{12}\times\frac{4}{11}\times\frac{4}{10}=\frac{8}{165}$\\
	& JQK & $\frac{4}{12}\times\frac{4}{11}\times\frac{4}{10}=\frac{8}{165}$\\
	& QJK & $\frac{4}{12}\times\frac{4}{11}\times\frac{4}{10}=\frac{8}{165}$\\
	& KJQ & $\frac{4}{12}\times\frac{4}{11}\times\frac{4}{10}=\frac{8}{165}$\\
	& KQJ & $\frac{4}{12}\times\frac{4}{11}\times\frac{4}{10}=\frac{8}{165}$\\
	& QKJ & $\frac{4}{12}\times\frac{4}{11}\times\frac{4}{10}=\frac{8}{165}$\\
\hline
2J/1K & JJK & $\frac{4}{12}\times\frac{3}{11}\times\frac{4}{10}=\frac{2}{55}$\\
	& JKJ & $\frac{4}{12}\times\frac{4}{11}\times\frac{3}{10}=\frac{2}{55}$\\
	& KJJ & $\frac{4}{12}\times\frac{4}{11}\times\frac{3}{10}=\frac{2}{55}$\\
\hline
3J/1K & --- & 0\\
\hline
0J/2K & KKQ & $\frac{4}{12}\times\frac{3}{11}\times\frac{4}{10}=\frac{2}{55}$\\
	& KQK & $\frac{4}{12}\times\frac{4}{11}\times\frac{3}{10}=\frac{2}{55}$\\
	& QKK & $\frac{4}{12}\times\frac{4}{11}\times\frac{3}{10}=\frac{2}{55}$\\
\hline
1J/2K & KKJ & $\frac{4}{12}\times\frac{3}{11}\times\frac{4}{10}=\frac{2}{55}$\\
	& KJK & $\frac{4}{12}\times\frac{4}{11}\times\frac{3}{10}=\frac{2}{55}$\\
	& JKK & $\frac{4}{12}\times\frac{4}{11}\times\frac{3}{10}=\frac{2}{55}$\\
\hline
2J/2K & --- & 0\\
\hline
3J/2K & --- & 0\\
\hline
0J/3K & KKK & $\frac{4}{12}\times\frac{3}{11}\times\frac{2}{10}=\frac{1}{55}$\\
\hline
1J/3K & --- & 0\\
\hline
2J/3K & --- & 0\\
\hline
3J/3K & --- & 0\\
\end{tabular}\\
\begin{tabular}{|c|c|c c c c|}
\hline
   f(x,y) &&& x &&\\
\hline
   &   & 0 & 1 & 2 & 3\\
\hline
   & 0 & $\frac{1}{55}$ & $\frac{6}{55}$ & $\frac{6}{55}$ & $\frac{1}{55}$\\
   &&&&&\\
 y & 1 & $\frac{6}{55}$ & $\frac{16}{55}$ & $\frac{6}{55}$ & 0\\
   &&&&&\\
   & 2 & $\frac{6}{55}$ & $\frac{6}{55}$ & 0 & 0\\
   &&&&&\\
   & 3 & $\frac{1}{55}$ & 0 & 0 & 0\\
\hline
\end{tabular}
\end{center}
\item[\textbf{b.}] \textbf{$P[(X,Y)\in A]$, where $A$ is the region
$\{(x,y)\mid x+y \geq 2\}$.}\\
The values $(x,y)$ that satisfy this equation are 
(1,1),(1,2),(1,3),(2,0),(2,1),(2,2),(2,3),(3,0),(3,1),(3,2),(3,3).
Their values from the table are $\frac{16}{55},\frac{6}{55},\frac{6}{55},
\frac{6}{55},\frac{6}{55},\frac{1}{55},\frac{1}{55},0,0,0,0,0,0$. Just need
the sum of them.
\begin{center}
$P[(X,Y)\in A] = \frac{42}{55}$
\end{center}
\end{enumerate}

\paragraph{3.55 Given the joint density function}
\begin{eqnarray*}
f(x,y) = \left\{ \begin{array}{lll}
	\frac{6-x-y}{8}, & 0<x<2, & 2<y<4,\\
	0, & \mbox{elsewhere,}
	\end{array} \right.
\end{eqnarray*}
\textbf{find $P(1<Y<3 \mid X=1)$.}
\begin{eqnarray*}
P(1<Y<3 \mid X=1) = \int_2^3 g(y \mid X=1)dy
\end{eqnarray*}
Let $f_1(x)$ be the marginal of $X$.
\begin{eqnarray*}
f_1(x) & = & \int_{-\infty}^\infty f(x,y)dy\\
	& = & \int_2^4 \frac{6-x-y}{8}dy\\
	& = & \left. \frac{6-x}{8}y-\frac{y^2}{16} \right|_2^4\\
	& = & \left( \frac{4(6-x)}{8}-1\right)-
		\left(\frac{2(6-x)}{8}-\frac{1}{4}\right)\\
	& = & \frac{3-x}{4}\\
g(y\mid X=1) & = & \frac{f(x,y)}{f_1(x)}\\
	& = & \frac{\frac{6-1-y}{8}}{\frac{1}{2}}\\
	& = & \frac{5-y}{4}\\
P(1<Y<3\mid X=1) & = & \int_2^3 \frac{5-y}{4}dy\\
	& = & \left. \frac{5y}{4}-\frac{y^2}{8}\right|_2^3\\
	& = & \frac{15}{4}-\frac{9}{8}-\frac{10}{4}+\frac{4}{8}\\
	& = & \frac{5}{8}
\end{eqnarray*}

\end{document}
