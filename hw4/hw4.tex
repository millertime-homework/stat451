% STAT451 HW4

\documentclass{article}
\usepackage{anysize}
\usepackage{amsmath}
\usepackage{graphicx}

\marginsize{2cm}{2cm}{2cm}{2cm}

\title{STAT451 HW4}
\author{Russell Miller}
\date{\today}

\begin{document}

\maketitle

% HW Problems
% 3.2, 3.5, 3.7, 3.8, 3.12, and 3.23

\paragraph{3.2 An overseas shipment of 5 foreign automobiles contains 2 that 
have slight paint blemishes. If an agency receives 3 of these automobiles at 
random, list the elements of the sample space \emph{S} using the letters \emph{B} 
and \emph{N} for blemished and nonblemished, respectively; then to each 
sample point assign a value \emph{x} of the random variable \emph{X} 
representing the number of automobiles purchased by the agency with paint 
blemishes.\\}
Out of the 5 automobiles in each shipment, only 2 of them have blemishes.
This is every combination of 3 cars from 2 blemished and 3 nonblemished.
\begin{center}
\begin{tabular}{c|c c c c c c c}
S & BBN & BNB & NBB & BNN & NBN & NNB & NNN\\
\hline
x &  2  &  2  &  2  &  1  &  1  &  1  &  0
\end{tabular}
\end{center}

\paragraph{3.5 Determine the value \emph{c} so that each of the following 
functions can serve as a probability distribution of the discrete random 
variable \emph{X}:}
\begin{enumerate}
\item[a.] $f(x) = c(x^2 + 4)$, for $x = 0,1,2,3$\\
\\
The sum of $f(x)$ for all values of $x$, is 1.
\begin{eqnarray*}
  \sum\limits_{x=0}^3 c(x^2 + 4)  & = & 1\\
   c(4+5+8+13) & = & 1
\end{eqnarray*}
\begin{center}
$\boxed{c = \frac{1}{30}}$
\end{center}

\item[b.] $f(x) = c{2 \choose x}{3 \choose 3-x}$, for $x = 0,1,2$\\
\begin{eqnarray*}
  \sum\limits_{x=0}^2 c{2 \choose x}{3 \choose 3-x} & = & 1\\
  c{2 \choose 0}{3 \choose 3}+{2 \choose 1}{3 \choose 2}+
   {2 \choose 2}{3 \choose 1} & = & 1\\
  c(1)(1)+(2)(3)+(2)(3) & = & 1
\end{eqnarray*}
\begin{center}
$\boxed{c=\frac{1}{13}}$
\end{center}
\end{enumerate}

\pagebreak

\paragraph{3.7 The total number of hours, measured in units of 100 hours,
that a family runs a vacuum cleaner over a period of one year is a
continuous random variable $X$ that has the density function}
\begin{eqnarray*}
  f(x) = \left\{ \begin{array}{ll}
  	x, & 0 < x < 1,\\
	2-x, & 1 \leq x < 2,\\
	0, & \mbox{elsewhere.}
  \end{array}\right.
\end{eqnarray*}
\textbf{Find the probability that over a period of one year, a family 
runs their vacuum cleaner}
\begin{enumerate}
\item[a.] \textbf{less than 120 hours.}\\
\begin{eqnarray*}
  P(x < 1.2) & = & \int_0^1 xdx + \int_1^{1.2} (2-x)dx\\
             & = & .5 + .18
\end{eqnarray*}
\begin{center}
$\boxed{P(x <1.2) = .68}$
\end{center}

\item[b.] \textbf{between 50 and 100 hours.}\\
\begin{eqnarray*}
  P(.5 < x < 1) & = & \int_{.5}^1 xdx + \int_1^1 (2-x)dx\\
                  & = & .375 + 0
\end{eqnarray*}
\begin{center}
$\boxed{P(.5 < x < 1) = .375}$
\end{center}
\end{enumerate}

\paragraph{3.8 Find the probability distribution of the random variable 
$W$ in Exercise 3.3, assuming that the coin is biased so that a head is 
twice as likely to occur as a tail.\\}
The sample space of Example 3.3 was:
\begin{center}
\begin{tabular}{c|c c c c c c c c}
S & HHH & HHT & HTH & THH & HTT & TTH & THT & TTT\\
\hline
w &  3  &  1  &  1  &  1  &  -1 &  -1 &  -1 &  -3
\end{tabular}
\end{center}
The probability distribution was:
\begin{center}
\begin{tabular}{c|c c c c}
w    &  -3         & -1          & 1           & 3\\
\hline
f(w) & $\frac{1}{8}$ & $\frac{3}{8}$ & $\frac{3}{8}$ & $\frac{1}{8}$
\end{tabular}
\end{center}
But that's when all of those combinations were equally likely.
If heads is twice as likely as tails, then the probability of getting
a $HHH$ is $\frac{2}{3}^3 = \frac{8}{27}$. The probability of getting
a $HHT$ is $\left(\frac{2}{3}\right)\left(\frac{2}{3}\right)
  \left(\frac{1}{3}\right) = \frac{4}{27}$, and so on.\\
\\
The new
probability distribution looks like:
\begin{center}
\begin{tabular}{|c|c c c c|}
\hline
w    &  -3         & -1          & 1           & 3\\
\hline
f(w) & $\frac{8}{27}$ & $\frac{4}{9}$ & $\frac{2}{9}$ & $\frac{1}{27}$\\
\hline
\end{tabular}
\end{center}

\paragraph{3.12 }

\end{document}
