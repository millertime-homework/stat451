% STAT 451 HW2 
% Russell Miller
% Oct 13 2011

\documentclass{article}
\usepackage{amsmath}
\usepackage{amssymb}
\usepackage{anysize}
%\usepackage{wasysym}
\usepackage{graphicx}
%for upvarsigma
\usepackage{upgreek}

\marginsize{2cm}{2cm}{2cm}{2cm}

\title{STAT451 HW2}
\author{Russell Miller}
\date{\today}

\begin{document}

\maketitle

\paragraph{1. A student selected from a class will either be a boy or a girl. 
If the probability that a boy will be selected is .3, what is the probability
that a girl will be selected?\\}
Let the number of boys be 3, and the total class size be 10. The P(boy)=.3
would be satisfied, and there would be 7 girls remaining.
\begin{center}
$\boxed{P(girl)=.7}$
\end{center}

\paragraph{3. If the probability that student A will fail a certain statistics
examination is 0.5, the probability that student B will fail the examination
is 0.2, and the probability that both student A and student B will fail the
examination is 0.1, what is the probability that at least one of these two
students will fail the examination?\\}
$P(A)=.5$, $P(B)=.2$, $P(A\cap B)=.1$\\
What we're looking for is when A will fail \emph{or} B will fail.
This could be represented as $P(A \cup B)$.\\
By Theorem 6 we know that $P(A \cup B) = P(A) + P(B) - P(A \cap B)$.
\begin{center}
$\boxed{P(A \cup B) = .5+.2-.1=.6}$
\end{center}

\paragraph{6. Consider two events A and B such that $P(A)=\frac{1}{3}$
and $P(B)=\frac{1}{2}$. Determine the value of $P(B \cap A')$ for each of
the following conditions:}
\begin{enumerate}
\item[a.] \textbf{A and B are disjoint}\\
$P(B)$ does not share any outcomes with $P(A)$, so $P(B)$ is a subset
of $P(A')$. The intersection of $P(B)$ with $P(A')$ is just $P(B)$.
\begin{center}
$\boxed{P(B \cap A') = P(B) = \frac{1}{2}}$
\end{center}

\item[b.] \textbf{A $\subset$ B}\\
This is the donut hole problem. If you want all of $P(B)$ except for the $P(A)$
part:
\begin{center}
$\boxed{P(B \cap A') = P(B)-P(A)=\frac{1}{2}-\frac{1}{3}=\frac{1}{6}}$
\end{center}

\item[c.] \textbf{P(A $\cap$ B) = $\frac{1}{8}$}\\
Now we know they are not disjoint. The overlap section of the venn diagram
is the $\frac{1}{8}$ part. So in order to make sure we get the $P(B)$ without
the $P(A)$, we have to take:
\begin{center}
$\boxed{P(B \cap A') = P(B) - P(A \cap B) = \frac{1}{2}-\frac{1}{8}=\frac{3}{8}}$
\end{center}
\end{enumerate}

\paragraph{9. Prove that for any two events A and B, the probability that exactly
one of the two events will occur is given by the expression:}
\begin{center}
\textbf{P(A) + P(B) - 2P(A $\cap$ B)}
\end{center}
We want to know the probability that A will occur and not B,
or that B will occur and not A. These can be represented as $P(A \cap B')$ and
$P(B \cap A')$ respectively.\\
We also know that $P(A \cap B') = P(A)-P(A \cap B)$ and
$P(B \cap A') = P(B)-P(A \cap B)$.\\
Thus:\\
$P(A \cap B') + P(B \cap A')$\\
$P(A)-P(A \cap B) + P(B) - P(A \cap B)$\\
$P(A) + P(B) - 2P(A \cap B) \blacksquare$

\paragraph{10. A point $(x, y)$ is to be selected from the square S containing
all points $(x, y)$ such that $0 \leq x \leq 1$ and $0 \leq y \leq 1$. Suppose that
the probability that the selected point will belong to any specified subset of S is
equal to the area of that subset. Find the probability of each of the following subsets:}
\begin{enumerate}
\item[a.] $(x-\frac{1}{2})^2 + (y-\frac{1}{2})^2 \geq \frac{1}{4}$\\
This equation is the area above a semicircle within S. We can find this area by:
\begin{center}
$Area\; of\; square - Area\; of\; circle \over 2$\\
$\boxed{\frac{1-\pi\frac{1}{2}^2}{2} = .1073}$
\end{center}

\item[b.] $\frac{1}{2} < x + y < \frac{3}{2}$\\
This is the area between two right triangles, of height $\frac{1}{2}$ and width
$\frac{1}{2}$.
\begin{center}
$Area\; of\; square - 2(Area\; of\; triangle)$\\
$\boxed{1-2\left(\frac{1}{2}\left(\frac{1}{2}\right)^2\right) = \frac{3}{4}}$
\end{center}

\item[c.] $y \leq 1-x^2$\\
To get the area under this curve, simply integrate from 0 to 1.
\begin{center}
$\boxed{\int_0^1(1-x^2)dx = \frac{2}{3}}$
\end{center}

\item[d.] $x=y$\\
This is a line connecting one corner to another. If the square has 10000 possible 
points, that is 100x100. Then for each of the 100 possible values of x, only one 
matches y. So there would be one for each value of y. Meaning 
$\frac{100}{10000}$ or $\frac{1}{100}$.\\
If, however, there were only 100 possible points, that is 10x10. Then for each
of the 10 possible values of x, only one will match y. But this is true for each of
the 10 values of y, so it is $\frac{10}{100}$ or $\frac{1}{10}$.\\
So it seems it would depend on the number of possible outcomes.\\
\end{enumerate}

\begin{center}
$\upvarsigma$\_$\upvarsigma$
\end{center}

\pagebreak

\paragraph{Text 2.79 A random sample of 200 adults are classified by sex and
their level of education attained. If a person is picked at random from this group,
find the probability that:\\}
\begin{enumerate}
\item[a.] the person is a male, given that the person has a secondary education.\\
78 people have a secondary education, from that there are 28 males.
\begin{center}
$\boxed{\frac{28}{200} = .14}$
\end{center}

\item[b.] the person does not have a college degree, given that the person is a female.\\
There are 112 females. Of them 95 do not have a degree.
\begin{center}
$\boxed{\frac{95}{200} = .475}$
\end{center}
\end{enumerate}

\paragraph{Text 2.86 For married couples living in a certain suburb the probability 
that the husband will vote on a bond referendum is 0.21, the probability that his wife
will vote on the referendum is 0.28, and the probability that both the husband and the
wife will vote is 0.15. What is the probability that:\\}
\begin{enumerate}
\item[a.] at least one member of a married couple will vote\\
Let $P(h)=0.21$ be the probability the husband will vote. Let $P(w)=0.28$ be the
probability the wife will vote. Then, $P(h \cap w) = 0.15$ is the probability that both
will vote. $P(h \cup w)$ is the probability that at least one will vote. 
\begin{center}
$\boxed{P(h \cup w) = P(h) + P(w) - P(h \cap w) = 0.21+0.28-0.15 = 0.34}$
\end{center}

\item[b.] a wife will vote, given that her husband will vote\\
By the division rule we know that $P(w\mid h) = \frac{P(w \cap h)}{P(h)}$.
\begin{center}
$\boxed{P(w\mid h) = \frac{0.15}{0.21} = 0.7143...}$
\end{center}

\item[c.] a husband will vote, given that his wife will not vote\\
We can see that $P(w') = 1-P(w) = 1-0.28 = 0.72$\\
Also, $P(h \cap w') = P(h) - P(h \cap w) = 0.21 - 0.15 = 0.06$
Thus:
\begin{center}
$\boxed{P(h\mid w') = \frac{P(h \cap w')}{P(w')} = \frac{0.06}{0.72} = 0.08\bar3}$
\end{center}
\end{enumerate}

\paragraph{Did not finish the last 3 questions that were assigned from the book.}

\end{document}
