% STAT 451 HW3 
% Russell Miller

\documentclass{article}
\usepackage{amsmath}
\usepackage{amssymb}
\usepackage{anysize}
%\usepackage{wasysym}
\usepackage{graphicx}
%for upvarsigma
\usepackage{upgreek}

\marginsize{2cm}{2cm}{2cm}{2cm}

\title{STAT451 HW3}
\author{Russell Miller}
\date{\today}

\begin{document}

\maketitle

\paragraph{2.25 A certain shoe comes in 5 different styles with each style available 
in 4 distinct colors. If the store wishes to display pairs of these shoes 
showing all of its various styles and colors, how many different pairs would 
the store have on display?\\}
Well there are 20 different shoe designs. If each pair is to match, then there would
be 20 pairs on display. However, the pairs don't need to match in which case you
can use a different shoe for the left and right. You would only need to display
10 pairs, still showing all 20 designs.
\begin{center}
$\boxed{10}$
\end{center}

\paragraph{2.28 A drug for the relief of asthma can be purchased from 5 different 
manufacturers in liquid, tablet, or capsule form, all of which come in regular and 
extra strength. How many different ways can a doctor prescribe the drug for a 
patient suffering from asthma?}
\begin{center}
$\boxed{5 \times 3 \times 2 = 30}$
\end{center}

\paragraph{2.32}
\begin{enumerate}
\item[a.] \textbf{How many distinct permutations can be made from the letters of the word
\emph{columns}?}
\begin{center}
$\boxed{7! = 5040}$
\end{center}
\item[b.] \textbf{How many of these permutations start with the letter \emph{m}?}\\
Simply place the m in the first position, now 6-letter permutation.
\begin{center}
$\boxed{6! = 720}$
\end{center}
\end{enumerate}

\paragraph{2.34}
\begin{enumerate}
\item[a.] \textbf{In how many ways can 6 people be lined up to get on a bus?}\\
\begin{center}
$\boxed{6! = 720}$
\end{center}
\item[b.] \textbf{If 3 specific persons, among 6, insist on following each other, how many 
ways are possible?}\\
Now we just group those 3 together as one person, leaving 3 others for a total of 4.
\begin{center}
$\boxed{4! = 24}$
\end{center}
\item[c.] \textbf{If 2 specific persons, among 6, refuse to follow each other, how many 
ways are possible?}\\
There are 30 ways to position 2 of the six people, and the following would be positions
where one is following the other:
\begin{verbatim}
{1,2},{2,3},{3,4},{4,5},{5,6}
\end{verbatim}
So there are 25 ways to position the 2 specific persons that refuse to follow each other.
There are another 25 if you mirror each of those positions.
There are 4 remaining people, so:
\begin{center}
$\boxed{50 \times 4! = 1200}$
\end{center}
\end{enumerate}

\paragraph{2.37 In how many ways can 4 boys and 5 girls sit in a row if the boys and
girls must alternate?\\}
There is only one possible pattern:
\begin{verbatim}
GBGBGBGBG
\end{verbatim}
So simply arrange the 4 boys first ($4!$), then the 5 girls ($5!$), by sticking to that pattern.
\begin{center}
$\boxed{(4!)(5!) = 2880}$
\end{center}

\paragraph{2.39 In a regional spelling bee, the 8 finalists consists of 3 boys and 5 girls. 
Find the number of sample points in the sample space \emph{S} for the number of 
possible orders at the conclusion of the contest for:}
\begin{enumerate}
\item[a.] \textbf{all 8 finalists.}\\
To arrange 8 finalists in every possible order:
\begin{center}
$\boxed{8! = 40320}$
\end{center}
\item[b.] \textbf{the first 3 positions.}\\
To arrange 3 people from a total of 8 possible people:
\begin{center}
$\boxed{_8P_3 = 336}$
\end{center}
\end{enumerate}

\paragraph{2.101 In a certain region of the country it is known from past experience that 
the probability of selecting an adult over 40 years of age with cancer is 0.05. If the 
probability of a doctor correctly diagnosing a person with cancer as having the disease 
is 0.78 and the probability of incorrectly diagnosing a person without cancer as having 
the disease is 0.06, what is the probability that a person is diagnosed as having 
cancer?\\}
Let $P(C)$ be the probability of selecting an adult over 40 years of age with cancer.
Let $P(D)$ be the probability that they are diagnosed with having the disease.
We know the following:\\
$P(C) = 0.05$, $P(D \mid C) = 0.78$, $P(D \mid C') = 0.06$ and $P(C') = 1 - P(C) = 0.95$.\\
Thus by Bayes' Rule:
\begin{center}
$\boxed{P(D) = P(C)P(D \mid C) + P(C')P(D \mid C') = (0.05)(0.78)+(0.95)(0.06) = 0.096}$
\end{center}

\paragraph{2.103 Referring to Exercise 2.101, what is the probability that a person 
diagnosed as having cancer actually has the disease?\\}
\begin{center}
$\boxed{P(C \mid D) = \frac{P(C)P(D \mid C)}{P(D)} = \frac{(0.05)(0.78)}{0.096} = 0.40625}$
\end{center}

\pagebreak

\paragraph{2.105 Suppose that the four inspectors at a film factory are supposed to stamp 
the expiration date on each package of film at the end of the assembly line. John, who 
stamps 20\% of the packages, fails to stamp the expiration date once in every 200 
packages; Tom, who stamps 60\% of the packages, fails to stamp the expiration date once 
in every 100 packages; Jeff, who stamps 15\% of the packages, fails to stamp the expiration 
date once in every 90 packages; and Pat, who stamps 5\% of the packages, fails to stamp 
the expiration date once in every 200 packages. If a customer complains that her package 
of film does not show the expiration date, what is the probability that it was inspected by 
John?\\}
Let $P(I_1) = 0.2$ be the probability that John inspected a package, and $P(I_2) = 0.6$ be 
Tom, $P(I_3) = 0.15$ be Jeff, and $P(I_4) = 0.05$ be Pat.
Let $P(E)$ be the probability that the expiration date was not stamped.\\
So far we know:\\
$P(E \mid I_1) = \frac{1}{200}$, $P(E \mid I_2) = \frac{1}{100}$, 
$P(E \mid I_3) = \frac{1}{90}$, $P(E \mid I_4) = \frac{1}{200}$.\\
And we know we are looking for $P(I_2 \mid E)$.
\begin{center}
$P(E) = P(I_1)P(E \mid I_1) + P(I_2)P(E \mid I_2) + P(I_3)P(E \mid I_3) + P(I_4)P(E \mid I_4) = 
\frac{0.2}{200} + \frac{0.6}{100} + \frac{0.15}{90} + \frac{0.05}{200} = 0.008917$\\
$\boxed{P(I_2 \mid E) = \frac{P(I_2)P(E \mid I_2)}{P(E)} = \frac{(0.2)(0.005)}{0.008917} = 
0.1121495327}$
\end{center}

\paragraph{2.107 Pollution of the rivers in the United States has been a problem for many 
years. Consider the following events:\\
A = \{The river is polluted\}\\
B = \{A sample of water tested detects pollution.\}\\
C = \{Fishing permitted.\}\\
Assume $P(A) = 0.3$, $P(B \mid A) = 0.75$, $P(B \mid A') = 0.20$, $P(C \mid A \cap B) = 0.20$,
$P(C \mid A' \cap B) = 0.15$, $P(C \mid A \cap B') = 0.80$, and $P(C \mid A' \cap B') = 0.90$.}
\begin{enumerate}
\item[a.] \textbf{Find $P(A \cap B \cap C)$.}\\
With the following formula:
\begin{center}
$P(B \mid A) = \frac{P(B \cap A)}{P(A)}$\\
$P(A \cap B) = P(B \mid A)P(A) = (0.75)(0.3) = 0.225$ 
\end{center}
And the same formula again:
\begin{center}
$\boxed{P(A \cap B \cap C) = P(C \mid A \cap B)P(A \cap B) = (0.2)(0.225) = 0.045}$
\end{center}
\item[b.] \textbf{Find $P(B' \cap C)$.}\\
I'm just not seeing it.
\item[c.] \textbf{Find $P(C)$.}\\
In order to get $P(C)$, it needs to be understood that $(A \cap B)$, $(A \cap B')$, $(A' \cap B)$,
and $(A' \cap B')$ are disjoint and their sum is 1. Since we know $P(C \mid A \cap B)$, 
$P(C \mid A \cap B')$, $P(C \mid A' \cap B)$, and $P(C \mid A' \cap B')$ we can use Bayes'
Rule:\\
$P(C) = P(A \cap B)P(C \mid A \cap B) + P(A \cap B')P(C \mid A \cap B') +
P(A' \cap B)P(C \mid A' \cap B) + P(A' \cap B')P(C \mid A' \cap B')$\\
We know $P(A \cap B) = .225$ from earlier. Next we must find $P(A \cap B') = 
P(A) - P(A \cap B) = 0.075$.\\
Now we need to know $P(B \cap A') = P(B \mid A')P(A')$.\\
It is given that $P(B \mid A') = 0.20$ and we know that $P(A') = 1-P(A) = 0.7$. So 
$P(B \cap A') = 0.14$.\\ We can find $P(A' \cap B') = P(A')-P(A' \cap B) = 0.56$.\\
Now we can plug that all back into Bayes' Rule:
\begin{center}
$\boxed{P(C) = (0.225)(0.20)+(0.075)(0.80)+(0.14)(0.15)+(0.56)(0.90) = 0.63}$
\end{center}
\end{enumerate}

\end{document}
